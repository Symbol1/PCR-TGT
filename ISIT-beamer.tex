% !TEX encoding = UTF-8 Unicode
% !TEX program = lualatex
% !TEX spellcheck = English
% !TEX pdfSinglePage



\catcode`激13 \def激#1{\lccode`~`#1\lowercase{\catcode`#113\def~}}

\documentclass[12pt,aspectratio=169]{beamer}
	\beamertemplatenavigationsymbolsempty
	\setbeamercovered{transparent}
	\setbeamersize{text margin left=3mm,text margin right=3mm}
	\overfullrule1em
	\def\pp{\pause\par}
	\advance\parskip\fill
	\def\BCMH{\gdef\beamer@currentmode{handout}}
	\def\BCMB{\gdef\beamer@currentmode{beamer}}
	激色#1♯#2 {\definecolor{#1}{HTML}{#2}}
	色PMS2767♯182B49   色PMS3015♯00629B   色PMS1245♯C69214   色PMS116♯FFCD00     
	色PMS3115♯00C6D7   色PMS7490♯6E963B   色PMS3945♯F3E500   色PMS144♯FC8900     
	色Black♯000000     色CoolGray9♯747678 色PMS401♯B6B1A9    色Metallic871♯84754E
	\setbeamercolor{normal text}{bg=PMS3015,fg=white}
	\setbeamercolor{structure}{fg=PMS116}
	\setbeamercolor{alerted text}{fg=PMS144}
	\setbeamercolor{example text}{fg=PMS7490}

\usepackage{xurl}
	\hypersetup{pdfsubject=q-bio.QM; 15A78}

\usepackage{mathtools,unicode-math,emoji}
	\setmainfont{SourceSansPro-Light}
	\setsansfont{SourceSansPro-Light}
	\setmonofont{SourceSansPro-Light}
	\setemojifont{Apple Color Emoji}
	\setmathfont[sans-style=literal]{texgyrepagella-math.otf}
	激㏒{\log} 激㏑{\ln} 激√{\sqrt} 激÷{\frac} 激†#1†{\text{#1}}
	\def\bma#1{\begin{bmatrix}#1\end{bmatrix}}

\usepackage{tikz,tikz-cd}
	\usetikzlibrary{backgrounds}
	% https://tex.stackexchange.com/q/420034/
	\pgfmathdeclarefunction*{axis_height}0{\begingroup\pgfmathreturn.25em\endgroup}
	\pgfmathdeclarefunction*{rule_thickness}0{\begingroup\pgfmathreturn.06em\endgroup}
	\pgfmathdeclarefunction*{lw}0{\begingroup\pgfmathreturn.06em\endgroup}
	\tikzset{
		every picture/.style={cap=round, join=round, line width=rule_thickness},
		% https://tex.stackexchange.com/q/146908/
		alt/.code args={<#1>#2#3}{\alt<#1>{\pgfkeysalso{#2}}{\pgfkeysalso{#3}}},
		uncover/.style={alt=#1{}{opacity=.15}},only/.style={alt=#1{}{opacity=0}}
	}

\usepackage{pgfplotstable,booktabs,colortbl}
	\pgfplotsset{
		compat/show suggested version=false,compat=1.17,
		width=10cm, height=7.5cm
	}

\title{PCR, Tropical Arithmetic, and Group Testing}
\author[Wang--Gabrys--Vardy]{Hsin-Po Wang \texorpdfstring\\{}
                             with Ryan Gabrys and Alexander Vardy}
\institute{Department of Electrical and Computer Engineering,
           University of California San Diego}
%\date{ISIT ~ 2022-06-28 ~ Espoo}
\date{}
\titlegraphic{
	\vskip-1cm
	\begin{tikzpicture} [nodes={align=center}]
		\path (-3.5,0)
			node [above] {\includegraphics[height=2cm]{bear2.jpg}}
			node [below] {Slides\\\url{https://h-p.wang/isit}};
		\path (3.5,0)
			node [above] {\includegraphics[width=2cm]{library2.jpg}}
			node [below] {Preprint\\arXiv: 2201.05440}	;
	\end{tikzpicture}
}

\begin{document}
\makeatletter

\def\linkfil#1{\vbox to#1cm{\hbox{|}\vfil\hbox to3mm{\hfil|}}}
\defbeamertemplate*{sidebar left}{thinbold}{
	\pgfsetfillopacity0
	\hyperlinkframeendprev{\linkfil1}
	\vfil
	\hyperlinkslideprev{\linkfil3}
	\vfil
	\hyperlinkslidenext{\linkfil3}
	\vfil
	\hyperlinkframestartnext{\linkfil1}
	\vfilneg
	\pgfsetfillopacity1
}

\begin{frame}
	\maketitle
\end{frame}

\defbeamertemplate*{sidebar right}{thinbold}{
	\begin{tikzpicture} [overlay, x=3mm, y=\paperheight]
		\scriptsize
		\pgfmathsetmacro\overlayfrac{
			\insertoverlaynumber/(\insertframeendpage+1-\insertframestartpage)}
		\path[save path=\stare, yscale={1/max(\insertmainframenumber-1,1)}]
			(0,0) -| (-1,1-\insertframenumber) -| +(\overlayfrac,1) -| cycle;
		\tikzset{short/.pic={\node at (-.55,-.5) [rotate=-90]
			{\beamer@shorttitle\kern2em\beamer@shortauthor};}}
		\pic [white] {short};
		\fill [use path=\stare, black, opacity=.15];
	\end{tikzpicture}
}

\begin{frame}\frametitle{Motivation of This Work}
	\begin{tikzpicture} [overlay]
		\path (12,-2) node {\includegraphics[width=6cm]{nose2.jpg}};
	\end{tikzpicture}
	Overall goal is to screen many people for covid \\
	(or for the next pandemic).
	\pp
	Antigen testing and antibody testing: \\
	Cheap and fast; but not too sensitive.
	\pp
	\alert{PCR} (polymerase chain reaction) testing: \\
	Sensitive but expensive and slow; \\
	keep track of variants (alpha, delta, omicron, etc).
	\pp
	\centering
	Q: How to combine PCR testing and \alert{Group Testing} (GT)?
\end{frame}

\begin{frame}\frametitle{Outline of This Talk}
	The working principle of PCR testing.
	\par
	Variants of group testing (GT).
	\par
	Our GT, called tropical GT.
\end{frame}

\begin{frame}\frametitle{Working Principle of PCR}
	\begin{tikzpicture} [overlay]
		\path
			(11.5,0) node (X) {\includegraphics[width=3.5cm]{sauna.jpg}}
			(X.east) node [rotate=90, below, align=center, font=\tiny]
			{https://www.craiyon.com/ \\ prompt: test tubes in sauna}
		;
	\end{tikzpicture}
	A PCR machine is a sauna room for test tubes, \\
	with three settings: cold, warm, and hot. \\
	~
	\par
	\uncover<2->{Cold = a {\color{PMS116}primer} and
	a {\color{PMS144}polymerase} stick to a single-stranded DNA.} \\
	\uncover<3->{Warm =
	the polymerase synthesizes the complement strand of the DNA.} \\
	\uncover<4->{Hot =
	a double-stranded DNA splits into two single-stranded DNAs.} \\
	\tikzset{
		single/.pic={
			\draw [scale=.1] foreach \bp in {-9,...,9} {(\bp,.2) |- +(1,-2)};
		},
		primer/.pic={
			\draw [scale=.1,PMS116] foreach \bp in {-9,...,-5} {(\bp,.2) |- +(-1,2)};
			\draw [scale=.1] foreach \bp in {-9,...,9} {(\bp,-.2) |- +(1,-2)};
			\draw [scale=.1,PMS144](-3,2)circle(2)node{P};
		},
		double/.pic={
			\draw [scale=.1] foreach \bp in {-9,...,9} {(\bp,.2) |- +(-1,2)}
			                 foreach \bp in {-9,...,9} {(\bp,-.2) |- +(1,-2)};
		}
	}
	\begin{tikzpicture}
		\draw [uncover=<2->] (0,0) pic {single};
		\draw [uncover=<2->] (2-.5,0) edge[->] +(1,0);
		\draw [uncover=<2->] (4,0) pic {primer};
		\draw [uncover=<3->] (6-.5,0) edge[->] +(1,0);
		\draw [uncover=<3->] (8,0) pic {double};
		\draw [uncover=<4->] (10-.5,0) edge[->] +(1,1) edge[->] +(1,-1);
		\draw [uncover=<4->] (12,1) pic[rotate=180]{single} (12,-1) pic{single};
	\end{tikzpicture}
\end{frame}

\begin{frame}\frametitle{How to Detect DNA and What's Ct Value?}
	\begin{tikzpicture}[overlay, shift={(10,-5)}]
		\draw [->] (0,-1) -- (0,4) node[left]{DNA\only<2->{/brightness}};
		\draw [->] (-1,0) -- (4,0) node[below]{cycle};
		\draw foreach \ct in {1,...,4}{
			plot [domain=0:\ct, samples=\ct+1] (\x, {2^(\x+2-\ct)}) }
			plot [domain=0:4, samples=5+1] (\x, {2^(\x+2-5)})
		;
		\draw[dotted, only=<3->] (0,3) -- (4,3)
			node[above, fill=PMS3015]{threshold};
	\end{tikzpicture}
	The amount of DNA \alert{doubles} every cold-warm-hot cycle.
	\pp
	Insert fluorescent dyes that like to attach to DNA. \\
	As the amount of DNA increases, the tube glows.
	\pp
	\alert{Ct} (cycle threshold) \alert{value} is \\
	\#cycles before we see the tube glowing.
\end{frame}

\begin{frame}\centering
	\huge\alert{So...}
	\par
	\Huge Can GT Make Good Use of Ct Values?
\end{frame}

\begin{frame}\frametitle{Review: Binary GT}
	In binary GT, a test result is either \alert{negative} or \alert{positive}.
	\par
	\begin{tikzpicture} [overlay, shift={(12,-1)}, rotate=-30]
		\draw (-.7,4) -- (-.7,0) arc [start angle=-180, end angle=0, radius=.7]
			-- (.7,4);
		\foreach \i in {0,...,4}{
			\draw [line width=lw*3] ({(\i-2)/5},{(-1)^\i/4}) -- +(0,.5);
			\draw ({(\i-2)/5},{(-1)^\i/4}) -- +(0,5);
		}
	\end{tikzpicture}
	Mix samples of five people. \\
	If the mixture is negative, all five people are healthy. \\
	If the mixture is positive, at least one is infected.
	\par
	Origin = [Dorfman 1943].  Book = [Du--Hwang 1993].
	Lecture note: [Ngo--Rudra 2011]. \\
	Recent survey: [Aldridge--Johnson--Scarlett 2019].
\end{frame}

\begin{frame}\frametitle{Review: Threshold GT}
	If less than $L$ people are infected, the mixture is negative. \\
	If more then $U$ people are infected, the mixture is positive. \\
	\alert{Inconclusive} if between $L$ and $U$.
	\par
	Binary GT: $(L,U) = (0,1)$.
	\par	
	[Damaschke 2006] [Dyachkov 2013] [Cheraghchi 2013]
\end{frame}

\begin{frame}\frametitle{Review: Quantitative GT}
	\begin{tikzpicture} [overlay]
		\path (12,-3) node {\includegraphics[width=6cm]{weigh2.jpg}};
	\end{tikzpicture}
	You have ten bags of coins, each containing many coins.
	Each coin weighs 5 grams. \\
	One bag contains fake coins; each fake coin weighs 4.5 grams. \\
	Task: Use a \alert{spring scale} to find the fake bag.
	\par
	Another name = coin-weighing problem. \\{}
	[Hwang 1987] [Guy--Nowakowski 1995] [Bshouty 2009]
\end{frame}

\begin{frame}\frametitle{Review: Semi-Quantitative GT}
	The spring scale is \alert{rusty}, accurate up to 1 gram.
	\par
	This version is basically a combination of quantitative GT and threshold GT.
	\par
	[Emad--Milenkovic 2014] [Cheraghchi--Gabrys--Milenkovic 2021]
\end{frame}

\begin{frame}\frametitle{Review: Compressed Sensing}
	\begin{tikzpicture} [overlay]
		\path (12,-2) node {\includegraphics[width=6cm]{scan2.jpg}};
	\end{tikzpicture}
	Very similar to semi-quantitative GT. \\
	Want to solve $𝐲 = A𝐱 + †errors†$.
	\par
	Some meta choices: \\
	$A$ is zero-one matrix or with real numbers? \\
	Usual matrix multiplication $(A·B)_{ik} ≔ ∑_j (A_{ij}·B_{jk})$ \\
	or logical version $(A∧B)_{ik} ≔ ⋁_j (A_{ij} ∧ B_{jk})$? \\
	Minimize $|A𝐱 - 𝐲|₂² + λ|𝐱|₁$ or other metric?
	\par
	Recent works: [Ghosh et al.\ 2021]
	[Shental et al.\ 2020] [Mutesa et al.\ 2021] \\
	Survey: [Aldridge--Ellis 2022] 
\end{frame}

\begin{frame}\centering
	\huge\alert{But...}
	\par
	\Huge Ct Values Do Not Fit.
\end{frame}

\begin{frame}\frametitle{PCR Precision Issue}
	\begin{tikzpicture}[overlay, shift={(10,-3)}]
		\draw [->] (0,-1) -- (0,4) node[left]{brightness};
		\draw [->] (-1,0) -- (4,0) node[below]{cycle};
		\draw foreach \ct in {1,...,4}{
			plot [domain=0:\ct, samples=\ct+1] (\x, {2^(\x+2-\ct)}) }
			plot [domain=0:4, samples=5+1] (\x, {2^(\x+2-5)})
		;
		\draw [dotted] (0,3) -- (4,3) node[above, fill=PMS3015]{threshold};
		\draw [PMS116, line width=2*lw, only=<2->]
			plot [domain=0:2.1, samples=2+1] (\x, {2^(\x+2-2.1)});
		\draw [PMS144, line width=2*lw, only=<2->]
			plot [domain=0:2.9, samples=3+1] (\x, {2^(\x+2-2.9)});
	\end{tikzpicture}
	DNA can double many many times. \\
	PCR is as sensitive as 100 copies/milliliter.
	\pp
	On the other hand, {\color{PMS144}$x$} copies/ml
	and {\color{PMS116}$1.9x$} copies/ml \\
	may have the same Ct value.
\end{frame}

\begin{frame}\frametitle{And the ``Problem'' with Logarithmic Scale}
	\raisebox{-12pt}{\fontsize{36pt}{0pt}\emoji{loudspeaker}}
	White noises of 50 dB and 30 dB combined = 50.043 dB.
	\par
	\hfill
	Mixing pH 1 and pH 3 acids = diluting pH 0.9957 by two-fold.
	\raisebox{-12pt}{\fontsize{36pt}{0pt}\emoji{alembic}}
	\par
	\raisebox{-12pt}{\fontsize{36pt}{0pt}\emoji{volcano}}
	Magnitude 9 and magnitude 8 earthquakes together = 9.009.
	\par
	\hfill
	Star with apparent magnitude 1 close to star with 6 = looks like 0.9892.
	\raisebox{-12pt}{\fontsize{36pt}{0pt}\emoji{milky-way}}
\end{frame}

\begin{frame}\centering
	\huge\alert{Actual Question is..}
	\par
	\Huge How to ``Add'' under Logarithmic Scale?
\end{frame}

\begin{frame}\frametitle{Use Tropical Arithmetics!}
	\begin{tikzpicture} [overlay]
		\path (12,-1) node {\includegraphics[width=6cm]{tropical2.jpg}};
	\end{tikzpicture}
	Rules are as follows: \\
	The domain is real numbers and infinity $ℝ ∪ \{∞\}$. \\
	Tropical addition: $x ⊕ y ≔ \min(x, y)$. \\
	Tropical multiplication $x ⊙ y ≔ x + y$.
	\par
	Hint: It's all about logarithm. \\
	$2^{-x} + 2^{-y} ≈ 2^{-\min(x, y)}$, especially when $|x-y|$ is big \\
	$2^{-x} · 2^{-y} = 2^{-(x + y)}$
\end{frame}

\begin{frame}\frametitle{Extend Tropical Arithmetics to Matrix Multiplication}
	Let $A ⊙ B$ be a matrix whose $(i,k)$th entry is let to be
	$⨁_j (A_{ij}⊙B_{jk}) = \min_j (A_{ij}+B_{jk})$.
	\pp
	Combinatorial meaning: \\
	Suppose $X₁, …, X_ℓ$, $Y₁, …, Y_m$, $Z₁, …, Z_n$ are points on Google map. \\
	Let the distance from $X_i$ to $Y_j$ be $A_{ij}$.   
	Let the distance from $Y_j$ to $Z_k$ be $B_{jk}$. \\
	$(A⊙B) _ {ik}$ is the distance from $X_i$ to $Z_k$
	via the best choice of $Y_j$. \\
	\centering
	\begin{tikzpicture}
		\draw [yscale=2/3, dotted]
			(0,2.5) node (X) {$X_i$}
			foreach \y in {1,...,5} { (3,\y) node(Y\y){$Y_\y$} }
			(6,2) node (Z) {$Z_k$}
			foreach \y in {1,...,5} { (X) -- (Y\y) -- (Z)}
		;
		\path (X) --node[above]{$A_{i5}$} (Y5) --node[above]{$B_{5k}$} (Z);
		\draw (X) -- (Y2) -- (Z);
	\end{tikzpicture}
\end{frame}

\begin{frame}\frametitle{Axiomize PCR and Pooling}
	Suppose there are $n$ samples with Ct values $x₁, x₂, …, x_N$. \\
	The Ct value of the mixture should be
	$-㏒₂ \Bigl( ∑_{j=1}^N 2^{-x_j} \Bigr)$.
	\pp
	This quantity is close to, and we \alert{pretend} that it is exactly,
	\[ \bma{0&0&⋯&0} ⊙ \bma{x₁\\x₂\\⋮\\x_N}
		= ⨁_{j=1}^N x_j = \min_{1≤j≤N} x_j. \]
\end{frame}

\begin{frame}\frametitle{Axiomize PCR and Pooling ... and Delay!}
	Suppose there are $n$ samples with Ct values $x₁, x₂, …, x_N$. \\
	Suppose we insert them into the PCR machine
	after $δ₁, δ₂, …, δ_N$ cycles, respectively. \\
	The final Ct value should be $-㏒₂ \Bigl( ∑_{j=1}^N 2^{-δ_j-x_j} \Bigr)$.
	\pp
	This is close to, and we \alert{pretend} that it's exactly,
	\[ \bma{δ₁&δ₂&⋯&δ_N} ⊙ \bma{x₁\\x₂\\⋮\\x_N}
		= ⨁_{j=1}^N (δ_j⊙x_j) = \min_{1≤j≤N} (δ_j+x_j). \]
\end{frame}

\begin{frame}\frametitle{Problem Statement}
	A \alert{$(T,N,D)$-tropical code} is a matrix
	$S ∈ (ℤ∪\{∞\}) ^ {T×N}$ such that, \\
	for any two column vectors $𝐱,𝐲 ∈ (ℤ∪\{∞\}) ^ {N×1}$,
	each with at most $D$ finite entries, \\
	$$ S⊙𝐱 ≠ S⊙𝐲. $$
	\pp
	A tropical code is said to be \alert{within maximum delay $ℓ$}
	if $S ∈ \{0,1,…,ℓ,∞\} ^ {T×N}$.
	\pp
	Goal: Find good tropical codes.
\end{frame}

\def\tubepop#1#2,{\xdef\car{#1}\xdef\cdr{#2}}
\def\eatubepop{\expandafter\tubepop}
\tikzset{
	syringe/.pic={
		\tikzset{scale=1/5,rotate=-30,name prefix/.get=\person}
		\draw(0,0)--(0,-3)node{\person};
		\draw(-2,6)-|(-1cm-10*lw,1)(1cm+10*lw,1)|-(2,6);
		\draw[line width=5*lw](-1,0)rectangle(1,2);
		\filldraw[PMS3015,line width=3*lw](-1,0)rectangle(1,3);
		\fill[\person](-1,0)rectangle(1,3);
		\draw(-1,9)--(1,9)(-1/2,9)|-(-1,3cm+10*lw)--+(2,0)-|(1/2,9);
		\draw(0,3)node()[transform shape,inner xsep=2cm,inner ysep=6cm]{};
	},
	tube/.pic={
		\begin{scope}[transparency group,scale=1/5,rotate=-30]
			\draw[.](-1cm-10*lw,1)--+(0,7)(1cm+10*lw,1)--+(0,7);
			\draw[.,line width=5*lw](-1,2)--(-1,1/2)--(0,-1/2)--(1,1/2)--(1,2)
				node[right,name prefix/.get=\tube]{\eatubepop\tube,\car};
			\filldraw[.,PMS3015,line width=3*lw]
				(-1,3)--(-1,1/2)--(0,-1/2)--(1,1/2)--(1,3);
			\eatubepop\tubecontentstring{},
			\ifx\car\pgfutil@empty\else
				\fill[PMS3015,\car](-1,1)--(-1,1/2)--(0,-1/2)--(1,1/2)|-cycle;
			\fi
			\foreach\y in{1,...,7}{
				\eatubepop\cdr{},
				\ifx\car\pgfutil@empty\else
					\fill[PMS3015,\car](-1,\y)rectangle+(2,1);
				\fi
			}
			\draw(0,4)node()[transform shape,inner xsep=2cm,inner ysep=5cm]{};
		\end{scope}
	},
	X/.style={PMS116},
	Y/.style={PMS7490},
	Z/.style={PMS144},
	T/.code={\def\tubecontentstring{#1}},
}

\begin{frame}\centering
	\huge\alert{Why Delay?}
	\par
	\Huge How Does Delaying Help GT?
\end{frame}

\begin{frame}
	\begin{tikzpicture}	
		\draw(0,4)pic(X){syringe}(0,2)pic(Y){syringe}(0,0)pic(Z){syringe};
		\draw[uncover=<2->]
			(5,3) pic(B1)[T=Z]{tube}
			(5,1) pic(A1)[T=X]{tube}
			(Z) edge[->,'] node[sloped,auto]{add} (B1)
			(X) edge[->]   node[sloped,auto]{add} (A1)
		;
		\draw[only=<3->]
			(9,3) pic(B2)[T=ZZ]{tube}
			(9,1) pic(A2)[T=XX]{tube}
			(B1) edge[->>]node[auto,']{dup} (B2)
			(A1) edge[->>]node[auto]  {dup} (A2)
		;
		\draw[uncover=<4->]
			(9,3) pic(B2)[T=ZZY]{tube}
			(9,1) pic(A2)[T=XXY]{tube}
			(B1) edge[->>]node[auto,']{dup} (B2)
			(A1) edge[->>]node[auto]  {dup} (A2)
			(Y) edge[->,bend left=30]    node[sloped,auto]{add} (B2)
			(Y) edge[->,bend right=30,'] node[sloped,auto]{add} (A2)
		;
		\draw[only=<5->]
			(13,3) pic(B3)[T=ZZZZYY]{tube}
			(13,1) pic(A3)[T=XXXXYY]{tube}
			(B2) edge[->>]node[auto,']{dup} (B3)
			(A2) edge[->>]node[auto]  {dup} (A3)
		;
		\draw[uncover=<6->]
			(13,3) pic(B3)[T=ZZZZYYX]{tube}
			(13,1) pic(A3)[T=XXXXYYZ]{tube}
			(B2) edge[->>]node[auto,']{dup} (B3)
			(A2) edge[->>]node[auto]  {dup} (A3)
			(X) edge[->,bend left=20,'] node[sloped,auto]{add} (B3)
			(Z) edge[->,bend right=20]  node[sloped,auto]{add} (A3)
		;
	\end{tikzpicture}
	\\[-5mm]
	\[\begin{bmatrix}
		a \\ b
	\end{bmatrix}
	≔
	\begin{bmatrix}
		0 & 1 & 2 \\
		2 & 1 & 0
	\end{bmatrix}
	⊙
	\begin{bmatrix}
		x \\ y \\ z
	\end{bmatrix}
	=
	\begin{bmatrix}
		\min(\, 0+x\,,\, 1+y\,,\, 2+z \,) \\
		\min(\, 2+x\,,\, 1+y\,,\, 0+z \,)
	\end{bmatrix}\]
\end{frame}

\begin{frame}\frametitle{Decoding the Previous Slide}
	Suppose at most one person is infected.
	\par
	X is infected iff $a-b = -2$. \\
	Y is infected iff $a-b = 0$. \\
	Z is infected iff $a-b = 2$.   
	\begin{tikzpicture}[overlay, shift={(4,-1)}]
		\draw[->] (0,0) -- (0,5) node [left] {$b$};
		\draw[->] (0,0) -- (5,0) node [right] {$a$};
		\begin{scope} [line width=.7071cm, line cap=butt]
			\draw[X] (0,2) -- node[sloped, black] {X is sick} (3,5);
			\draw[Y] (0,0) -- node[sloped, black] {Y is sick} (4.5,4.5);
			\draw[Z] (2,0) -- node[sloped, black] {Z is sick} (5,3);
		\end{scope}
	\end{tikzpicture}
\end{frame}

\begin{frame}\frametitle{Main Results on Nonadaptive Tropical GT}
	When there is $D = 1$ infected person in a population of size $N$, \\
	and the delay is limited to $ℓ$ cycles,
	we will use$ T ≈ ㏒_{ℓ+1}(N)$ tests.
	\par
	When there are $D = 2$ infected persons in a population of size $N$: \\
	\begin{itemize}
		\item The first construction uses $T ≈ 2√N$ tests.  \\
		In this construction, every person is present in only two tests.
		\item The second construction uses $T ≈ 1.01 ㏒₂N$ tests \\
		and limits the delay to $ℓ ≈ 3㏒₂(N)$ cycles. \\
		This outperforms the IT bound of binary GT.
	\end{itemize}
	\par
	For general $D$, we give one necessary condition and two sufficient
	conditions.
\end{frame}

\begin{frame}\frametitle{Main Results on Adaptive Tropical GT}
	When adaptive testing is allowed, $T = 4$ tests are sufficient to find \\
	$D = 2$ infected persons among arbitrarily many persons.
	\par
	In general, $T = 3D+1$ tests are sufficient \\
	to locate $D$ infected persons among arbitrarily many persons. \\
	For this construction, one does not need to know $D$ beforehand.
	\par
	When delays are limited to $ℓ$ cycles,
	we show that $T ≈ 4D㏒_ℓ(N)$ tests suffice. \\
	For this construction, one does not need to know $D$ beforehand.
\end{frame}

\begin{frame}\frametitle{Summary of Novelty}
	\begin{tikzpicture} [overlay]
		\path (12,-2.5) node {\includegraphics[width=6cm]{virus2.jpg}};
	\end{tikzpicture}
	1. We use $x ⊕ y ≔ \min(x, y)$ to characterize the result \\
	   of mixing Ct values $x$ and $y$.  This simplifies decoding.
	\pp
	2. We use $δ ⊙ x ≔ δ + x$, i.e., delaying, to enhance GT. \\
	   This inspires new combinatorics problems.
	\pp
	3. Tropical matrix multiplication becomes \\
	   a succinct language.  Nonadaptive tropical GT \\
	   looks like ``tropical compressed sensing.''
\end{frame}

\begin{frame}\frametitle{Future Works \\ Open to Questions}
	• Random coding approach? Asymptotics?
	\par
	• Noisy/erroneous measurements?
	\par
	\centering
	\vskip5mm
	\inserttitlegraphic
	\vskip0pt plus-1fill
\end{frame}

\appendix

\begin{frame}\centering
	\huge\alert{Appendix}
	\par
	\Huge PCR Error Models
\end{frame}

\begin{frame}\frametitle{Error Models of PCR}
	Error model 1: Ct value is rounded to the nearest integer.
	\par
	Error model 2: Use fractional cycle counts (whatever that means). \\
	(When optimizing PCR for time,
	earlier cycles take more time and later cycles take less time.) \\
	Not all single-stranded DNA will be completed;
	DNA increases $1.9$-fold or $2$-fold.
	\par
	Error introduced independently:  Assume tropical addition.
\end{frame}

\begin{frame}\centering
	\huge\alert{Appendix}
	\par
	\Huge Simulation Plots
\end{frame}

\fontsize{11pt}{11pt}\selectfont
\newcount\accumulatenumplots
\tikzset{
	ROC/.style={
		overlay, shift={(-8.5,-6.5)}
	}
}
\pgfplotsset{
	every axis/.style={cycle list shift=\the\accumulatenumplots},
	cycle multiindex* list={
		PMS116,PMS7490,PMS1245,PMS3115,PMS144\nextlist
		mark=o,mark size=1.5\\mark=+\\mark=triangle\\mark=*,mark size=1\\mark=x\\
		every mark/.append style={yscale=-1},mark size=2,mark=triangle*\\\nextlist
	},
	legend image code/.code={
		\draw[mark repeat=2,mark phase=2,#1]plot coordinates{
			(-4pt,-6pt)(0pt,2pt)(8pt,6pt)
		};
	},
	RoC/.style={
		yticklabel=\pgfmathprintnumber\tick\%,
		xticklabel=\pgfmathprintnumber\tick\%,
		xticklabel style={rotate=90},
		xlabel=false positive rate ($1-{}$specificity),
		ylabel=true positive rate (sensitivity),
		xlabel near ticks,ylabel near ticks,
		y label style={inner sep=0pt},
		legend pos=south east,
		legend style={draw=white,fill=PMS3015}
	}
}

\pgfplotstableread{
   5fp   5sen   6fp   6sen   7fp   7sen   8fp   8sen   9fp   9sen   10fp  10sen
   0.463 99.323 0.805 98.907 1.273 98.394 1.873 97.793 2.627 97.088 3.531 96.314
   0.463 99.323 0.805 98.907 1.273 98.394 1.873 97.793 2.627 97.088 3.531 96.314
   0.464 99.325 0.808 98.91  1.278 98.4   1.879 97.801 2.636 97.099 3.545 96.329
   0.471 99.331 0.82  98.924 1.296 98.42  1.905 97.83  2.673 97.139 3.591 96.38 
   0.485 99.347 0.842 98.95  1.335 98.455 1.959 97.885 2.745 97.213 3.685 96.478
   0.51  99.367 0.883 98.989 1.397 98.513 2.048 97.97  2.865 97.33  3.843 96.637
   0.545 99.399 0.943 99.043 1.49  98.592 2.177 98.087 3.043 97.488 4.074 96.839
   0.596 99.443 1.027 99.109 1.611 98.695 2.351 98.228 3.277 97.686 4.376 97.102
   0.66  99.491 1.137 99.193 1.768 98.814 2.571 98.401 3.57  97.908 4.759 97.405
   0.741 99.545 1.269 99.286 1.967 98.955 2.839 98.594 3.924 98.167 5.214 97.737
   0.843 99.604 1.431 99.385 2.192 99.102 3.155 98.798 4.347 98.438 5.746 98.084
   0.961 99.668 1.619 99.488 2.46  99.263 3.518 99.003 4.824 98.719 6.344 98.444
   1.105 99.746 1.828 99.596 2.762 99.421 3.941 99.221 5.362 99.006 7.011 98.793
   1.256 99.821 2.069 99.703 3.11  99.575 4.406 99.435 5.955 99.286 7.747 99.131
   1.432 99.879 2.332 99.81  3.495 99.725 4.911 99.644 6.602 99.548 8.522 99.447
   1.634 99.944 2.635 99.899 3.922 99.865 5.474 99.821 7.307 99.784 9.365 99.728
   1.855 100    2.983 100    4.391 100    6.09  100    8.08  100   10.291 100   
}\tablepreval
\begin{frame}[t]
	\leftskip10cm
	\begin{tikzpicture}[ROC]
		\begin{axis}[RoC]
			\addplot table[x=5fp,y=5sen]{\tablepreval};\addlegendentry{$p=5\%$}
			\addplot table[x=6fp,y=6sen]{\tablepreval};\addlegendentry{$p=6\%$}
			\addplot table[x=7fp,y=7sen]{\tablepreval};\addlegendentry{$p=7\%$}
			\addplot table[x=8fp,y=8sen]{\tablepreval};\addlegendentry{$p=8\%$}
			\addplot table[x=9fp,y=9sen]{\tablepreval};\addlegendentry{$p=9\%$}
			\addplot table[x=10fp,y=10sen]{\tablepreval};\addlegendentry{$p=10\%$}
		\end{axis}
		\global\advance\accumulatenumplots6
	\end{tikzpicture}
	\par
	Assume uniform Ct values on the interval $[16,32]$, $15×35$ Kirkman triple
	system, and no delay ($ℓ=0$).  We vary the prevalence rate $p$ and plot the
	ROC curves.
\end{frame}

\pgfplotstableread{
	8fp    8sen    12fp   12sen   16fp   16sen   20fp   20sen   24fp   24sen
	3.291  96.837  3.391  96.608  3.538  96.338  3.745  95.816  4.111  94.889
	3.291  96.837  3.391  96.608  3.538  96.338  3.745  95.816  4.111  94.892
	3.295  96.841  3.398  96.616  3.55   96.354  3.774  95.856  4.204  95.04 
	3.311  96.855  3.422  96.643  3.597  96.404  3.876  95.979  4.513  95.462
	3.344  96.89   3.475  96.695  3.694  96.502  4.086  96.208  5.093  96.161
	3.396  96.941  3.564  96.776  3.854  96.66   4.418  96.54   5.972  97.082
	3.478  97.017  3.691  96.894  4.084  96.866  4.884  96.96   7.128  98.093
	3.584  97.11   3.865  97.048  4.387  97.122  5.482  97.454  8.557  99.071
	3.727  97.231  4.086  97.225  4.767  97.413  6.21   97.991 10.275 100.   
	3.899  97.364  4.357  97.437  5.223  97.737  7.062  98.547    nan     nan
	4.105  97.516  4.684  97.667  5.755  98.089  8.022  99.078    nan     nan
	4.344  97.685  5.057  97.914  6.35   98.442  9.089  99.547    nan     nan
	4.616  97.869  5.477  98.172  7.016  98.804 10.282 100.       nan     nan
	4.926  98.066  5.936  98.437  7.749  99.146    nan     nan    nan     nan
	5.265  98.268  6.447  98.706  8.54   99.474    nan     nan    nan     nan
	5.644  98.48   6.992  98.966  9.382  99.74     nan     nan    nan     nan
	6.055  98.682  7.578  99.211 10.299 100.       nan     nan    nan     nan
	6.491  98.884  8.197  99.439    nan     nan    nan     nan    nan     nan
	6.963  99.084  8.852  99.641    nan     nan    nan     nan    nan     nan
	7.457  99.273  9.544  99.836    nan     nan    nan     nan    nan     nan
	7.976  99.452 10.281 100.       nan     nan    nan     nan    nan     nan
	8.516  99.612    nan     nan    nan     nan    nan     nan    nan     nan
	9.081  99.753    nan     nan    nan     nan    nan     nan    nan     nan
	9.663  99.882    nan     nan    nan     nan    nan     nan    nan     nan
	10.283 100       nan     nan    nan     nan    nan     nan    nan     nan
}\tablerange
\begin{frame}[t]
	\leftskip10cm
	\begin{tikzpicture}[ROC]
		\begin{axis}[RoC, unbounded coords=jump]
			\addplot table[x=8fp,y=8sen]{\tablerange};\addlegendentry{$[8,32]$}
			\addplot table[x=12fp,y=12sen]{\tablerange};\addlegendentry{$[12,32]$}
			\addplot table[x=16fp,y=16sen]{\tablerange};\addlegendentry{$[16,32]$}
			\addplot table[x=20fp,y=20sen]{\tablerange};\addlegendentry{$[20,32]$}
			\addplot table[x=24fp,y=24sen]{\tablerange};\addlegendentry{$[24,32]$}
		\end{axis}
		\global\advance\accumulatenumplots5
	\end{tikzpicture}
	\par
	Assume prevalence rate $p = 10\%$, uniform Ct values, $15×35$ Kirkman triple
	system, and no delay ($ℓ = 0$).  We vary the range of the Ct values and plot
	the ROC curves.  Surprisingly, larger interval (consequently larger
	variance) is easier to decode.
\end{frame}

\pgfplotstableread{
	8fp    8sen    6fp    6sen    4fp    4sen    2fp    2sen    0fp    0sen
	1.988  97.433  2.326  97.014  2.725  96.596  3.139  96.2    3.549  96.339
	2.023  97.471  2.346  97.039  2.734  96.608  3.142  96.203  3.549  96.339
	2.083  97.535  2.388  97.085  2.763  96.639  3.156  96.221  3.563  96.355
	2.175  97.63   2.458  97.157  2.814  96.692  3.203  96.278  3.611  96.409
	2.303  97.744  2.562  97.259  2.895  96.773  3.3    96.379  3.709  96.508
	2.466  97.884  2.701  97.38   3.057  96.927  3.462  96.549  3.871  96.659
	2.664  98.047  2.873  97.532  3.287  97.134  3.693  96.774  4.104  96.87 
	2.902  98.229  3.18   97.771  3.592  97.385  4.003  97.058  4.409  97.124
	3.18   98.421  3.568  98.049  3.979  97.68   4.389  97.389  4.785  97.42 
	3.559  98.644  4.027  98.357  4.444  98.023  4.854  97.755  5.247  97.758
	3.987  98.87   4.574  98.682  4.983  98.386  5.391  98.139  5.773  98.098
	4.465  99.092  5.077  98.928  5.599  98.75   5.998  98.535  6.368  98.459
	4.989  99.308  5.621  99.155  6.29   99.113  6.676  98.926  7.036  98.814
	5.547  99.513  6.207  99.396  6.884  99.374  7.409  99.277  7.771  99.152
	6.147  99.696  6.823  99.619  7.509  99.613  8.205  99.605  8.554  99.468
	6.792  99.854  7.472  99.821  8.175  99.818  8.857  99.816  9.392  99.741
	7.474 100.     8.16  100.     8.86  100.     9.544 100.    10.306 100.   
}\tablelimit
\begin{frame}[t]
	\leftskip10cm
	\begin{tikzpicture}[ROC]
		\begin{axis}[RoC]
			\addplot table[x=8fp,y=8sen]{\tablelimit};\addlegendentry{$ℓ=8$}
			\addplot table[x=6fp,y=6sen]{\tablelimit};\addlegendentry{$ℓ=6$}
			\addplot table[x=4fp,y=4sen]{\tablelimit};\addlegendentry{$ℓ=4$}
			\addplot table[x=2fp,y=2sen]{\tablelimit};\addlegendentry{$ℓ=2$}
			\addplot table[x=0fp,y=0sen]{\tablelimit};\addlegendentry{$ℓ=0$}
		\end{axis}
		\global\advance\accumulatenumplots5
	\end{tikzpicture}
	\par
	Assume prevalence rate $p=10\%$, uniform Ct values on the interval
	$[16,32]$, $15×35$ Kirkman triple system, and $ℓ·†Bernoulli†(1/2)$ delay.
	We vary the limit of delay $ℓ$ and plot the ROC curves.
\end{frame}

\pgfplotstableread{
	Bfp    Bsen     Ufp    Usen     0fp    0sen
	1.99   97.42    2.4    96.751   3.543  96.327
	2.023  97.456   2.415  96.768   3.543  96.327
	2.082  97.519   2.446  96.803   3.556  96.342
	2.173  97.607   2.504  96.864   3.605  96.395
	2.298  97.718   2.601  96.967   3.704  96.496
	2.457  97.854   2.754  97.109   3.861  96.654
	2.659  98.017   2.974  97.3     4.093  96.866
	2.897  98.196   3.272  97.552   4.394  97.122
	3.172  98.386   3.653  97.843   4.772  97.425
	3.555  98.615   4.117  98.178   5.225  97.758
	3.985  98.846   4.646  98.52    5.756  98.116
	4.46   99.068   5.233  98.858   6.356  98.466
	4.977  99.284   5.848  99.173   7.026  98.823
	5.542  99.495   6.479  99.45    7.755  99.162
	6.15   99.682   7.095  99.685   8.527  99.471
	6.793  99.851   7.679  99.863   9.365  99.75 
	7.472 100.      8.193 100.     10.286 100.   
}\tabledistri
\begin{frame}[t]
	\leftskip10cm
	\begin{tikzpicture}[ROC]
		\begin{axis}[RoC]
			\addplot table[x=Bfp,y=Bsen]{\tabledistri};\addlegendentry{Bernoulli}
			\addplot table[x=Ufp,y=Usen]{\tabledistri};\addlegendentry{uniform}
			\addplot table[x=0fp,y=0sen]{\tablelimit};\addlegendentry{no delay}
		\end{axis}
		\global\advance\accumulatenumplots3
	\end{tikzpicture}
	\par
	Assume prevalence rate $p=10\%$, uniform Ct values on the interval
	$[16,32]$, $15×35$ Kirkman triple system, and $ℓ=8$.  We vary the
	distribution of the random delay $δ$ and plot the ROC curves.
\end{frame}

\pgfplotstableread{
	405fp  405sen   105fp  105sen   45fp   45sen    15fp   15sen
	2.588  96.606   2.602  96.61    2.787  96.555   3.534  96.319
	2.588  96.606   2.602  96.61    2.787  96.555   3.534  96.319
	2.601  96.618   2.616  96.622   2.8    96.567   3.547  96.333
	2.652  96.661   2.667  96.666   2.851  96.613   3.595  96.386
	2.759  96.747   2.774  96.75    2.955  96.7     3.695  96.484
	2.935  96.881   2.949  96.885   3.128  96.838   3.854  96.64 
	3.19   97.066   3.204  97.066   3.377  97.026   4.082  96.844
	3.53   97.3     3.544  97.295   3.713  97.261   4.385  97.107
	3.955  97.571   3.973  97.565   4.128  97.537   4.766  97.405
	4.471  97.879   4.49   97.87    4.634  97.839   5.227  97.738
	5.072  98.201   5.091  98.197   5.228  98.177   5.755  98.093
	5.76   98.537   5.776  98.531   5.892  98.515   6.354  98.447
	6.521  98.871   6.533  98.869   6.631  98.857   7.019  98.811
	7.356  99.192   7.368  99.192   7.44   99.181   7.747  99.141
	8.258  99.487   8.269  99.491   8.324  99.487   8.536  99.463
	9.23   99.751   9.237  99.754   9.27   99.746   9.372  99.74 
	10.284 100     10.288 100.     10.296 100.     10.292 100.   
}\tablekirkman
\begin{frame}[t]
	\leftskip10cm
	\begin{tikzpicture}[ROC]
		\begin{axis}[RoC]
			\addplot table[x=405fp,y=405sen]{\tablekirkman};\addlegendentry{$405×735$}
			\addplot table[x=105fp,y=105sen]{\tablekirkman};\addlegendentry{$105×315$}
			\addplot table[x=45fp,y=45sen]{\tablekirkman};\addlegendentry{$45×105$}
			\addplot table[x=15fp,y=15sen]{\tablekirkman};\addlegendentry{$15×35$}
		\end{axis}
		\global\advance\accumulatenumplots4
	\end{tikzpicture}
	\par
	Assume prevalence rate $p = 10\%$, uniform Ct values on $[16,32]$, and no
	delay ($ℓ = 0$).  We consider Kirkman triple systems of different size
	(after truncation so that the code rate $N/T = 7/3$ is fixed) and plot the
	ROC curves.
\end{frame}

\pgfplotstableread{
	T0fp    T0sen T1fp   T1sen  T2fp   T2sen  T3fp   T3sen  T4fp   T4sen  T5fp   T5sen  T6fp   T6sen  T7fp   T7sen  T8fp   T8sen  T9fp   T9sen
	1.826  97.984 1.809  97.96  1.823  98.022 1.871  98.007 1.837  97.99  1.857  98.007 1.842  98.054 1.852  98.005 1.866  98.008 1.851  97.983
	1.826  97.984 1.809  97.96  1.823  98.022 1.871  98.007 1.837  97.99  1.857  98.007 1.842  98.054 1.852  98.005 1.866  98.008 1.851  97.983
	1.834  97.99  1.816  97.965 1.833  98.031 1.882  98.011 1.846  97.996 1.869  98.014 1.851  98.056 1.861  98.01  1.875  98.012 1.861  97.989
	1.869  98.004 1.855  97.985 1.867  98.055 1.917  98.033 1.881  98.02  1.904  98.044 1.888  98.067 1.893  98.031 1.909  98.036 1.898  98.008
	1.951  98.04  1.925  98.034 1.949  98.097 1.999  98.082 1.957  98.061 1.978  98.087 1.957  98.122 1.963  98.076 1.979  98.082 1.972  98.058
	2.074  98.116 2.057  98.108 2.08   98.168 2.131  98.146 2.084  98.12  2.114  98.145 2.089  98.185 2.093  98.128 2.112  98.158 2.103  98.128
	2.27   98.209 2.238  98.206 2.269  98.269 2.313  98.236 2.273  98.226 2.295  98.245 2.276  98.272 2.273  98.224 2.3    98.257 2.288  98.22 
	2.535  98.357 2.498  98.352 2.528  98.405 2.562  98.369 2.528  98.347 2.545  98.353 2.523  98.397 2.541  98.351 2.559  98.39  2.548  98.348
	2.857  98.494 2.823  98.521 2.865  98.551 2.894  98.533 2.848  98.509 2.888  98.51  2.86   98.554 2.874  98.507 2.889  98.518 2.878  98.493
	3.284  98.661 3.241  98.704 3.265  98.703 3.305  98.693 3.257  98.67  3.293  98.689 3.282  98.736 3.276  98.683 3.301  98.687 3.286  98.684
	3.761  98.875 3.737  98.906 3.737  98.897 3.785  98.865 3.734  98.85  3.773  98.879 3.767  98.933 3.761  98.873 3.779  98.834 3.776  98.868
	4.31   99.074 4.304  99.097 4.316  99.08  4.343  99.051 4.293  99.052 4.34   99.067 4.318  99.142 4.338  99.072 4.344  99.046 4.357  99.067
	4.964  99.283 4.916  99.3   4.962  99.273 4.98   99.284 4.94   99.235 4.978  99.267 4.98   99.322 4.967  99.263 4.981  99.248 4.992  99.241
	5.674  99.489 5.647  99.493 5.683  99.487 5.709  99.473 5.668  99.425 5.706  99.459 5.687  99.489 5.714  99.483 5.689  99.445 5.712  99.412
	6.481  99.671 6.432  99.702 6.473  99.658 6.525  99.649 6.454  99.607 6.492  99.645 6.482  99.657 6.534  99.66  6.457  99.648 6.513  99.617
	7.353  99.832 7.331  99.867 7.358  99.831 7.356  99.833 7.363  99.791 7.365  99.82  7.369  99.825 7.432  99.845 7.324  99.811 7.388  99.816
	8.331 100.    8.267 100.    8.347 100.    8.297 100.    8.33  100.    8.307 100.    8.315 100.    8.389 100.    8.281 100.    8.339 100.   
}\tabletapestry
\begin{frame}[t]
	\leftskip10cm
	\begin{tikzpicture}[ROC]
		\begin{axis}[RoC,every axis plot/.append style={line width=.1}]
			\draw(4.5,99.3)circle(.8pt)
				node[above left,align=right]{Tapestry \\ \& std devi}
				(4.5+2.41,99.3)--(4.5-2.41,99.3)
				(4.5,100)coordinate(T+y)(4.5,99.3-2.55)coordinate(T-y)
				(5,99)circle(.8pt)node[below right]{$(5\%,99\%)$};
			\addplot table[x=T0fp,y=T0sen]{\tabletapestry};
			\addplot table[x=T1fp,y=T1sen]{\tabletapestry};
			\addplot table[x=T2fp,y=T2sen]{\tabletapestry};
			\addplot table[x=T3fp,y=T3sen]{\tabletapestry};
			\addplot table[x=T4fp,y=T4sen]{\tabletapestry};
			\addplot table[x=T5fp,y=T5sen]{\tabletapestry};
			\addplot table[x=T6fp,y=T6sen]{\tabletapestry};
			\addplot table[x=T7fp,y=T7sen]{\tabletapestry};
			\addplot table[x=T8fp,y=T8sen]{\tabletapestry};
			\addplot table[x=T9fp,y=T9sen]{\tabletapestry};
		\end{axis}
		\draw(T+y)--(T-y);
		\global\advance\accumulatenumplots10
	\end{tikzpicture}
	\par
	Assume $D = 10$ patients within $N = 105$ persons (infection rate $9.52\%$),
	uniform Ct values on $[16,32]$, $45×105$ Kirkman triple system (truncation
	of a $45×330$ Kirkman triple system), and no delay ($ℓ = 0$).  We plot $10$
	ROC curves.  Each curve is $10{,}000$ encoding--decodings, i.e., $450{,}000$
	tubes, $100{,}000$ patients, and $1{,}050{,}000$ test takers.  Compare this
	to Tapestry's data point and its standard deviations $(4.50\% ± 2.41\%,
	99.30\% ± 2.55\%)$ (Table S.XII of the preprint version
	[Ghosh et al.\ 2020]).
\end{frame}

\pgfplotstableread{
	97fp   97sen    49fp   49sen    17fp   17sen    Pfp    Psen
	0.892  96.809   1.015  96.099   1.862  94.293   2.9    96.004
	0.892  96.809   1.015  96.099   1.862  94.303   2.9    96.004
	0.893  96.809   1.022  96.108   1.922  94.424   2.901  96.005
	0.899  96.815   1.045  96.141   2.037  94.625   2.907  96.013
	0.916  96.832   1.095  96.204   2.203  94.906   2.927  96.037
	0.953  96.869   1.179  96.31    2.416  95.242   2.972  96.091
	1.018  96.934   1.301  96.462   2.671  95.626   3.056  96.191
	1.12   97.037   1.469  96.661   2.963  96.066   3.193  96.347
	1.268  97.179   1.682  96.92    3.297  96.515   3.395  96.565
	1.468  97.378   1.944  97.211   3.659  96.973   3.672  96.852
	1.727  97.629   2.254  97.55    4.047  97.436   4.039  97.206
	2.047  97.936   2.619  97.937   4.463  97.907   4.501  97.631
	2.436  98.294   3.025  98.338   4.9    98.355   5.065  98.096
	2.891  98.69    3.482  98.76    5.362  98.806   5.721  98.588
	3.414  99.12    3.988  99.176   5.834  99.233   6.482  99.093
	4.008  99.557   4.536  99.596   6.327  99.626   7.334  99.561
	4.69  100.      5.135 100.      6.837 100.      8.296 100.   
}\tabelpbest
\begin{frame}[t]
	\leftskip10cm
	\begin{tikzpicture}[ROC]
		\begin{axis}[RoC]
			\addplot table[x=97fp,y=97sen]{\tabelpbest};\addlegendentry{$(2,4,97)$-S}
			\addplot table[x=49fp,y=49sen]{\tabelpbest};\addlegendentry{$(2,3,49)$-S}
			\addplot table[x=17fp,y=17sen]{\tabelpbest};\addlegendentry{$K_{17}$}
			\addplot table[x=Pfp,y=Psen]{\tabelpbest};\addlegendentry{P-Best}
		\end{axis}
		\global\advance\accumulatenumplots4
	\end{tikzpicture}
	\par
	Assume prevalence rate $p = 2\%$, uniform Ct values on $[16,32]$, and no
	delay ($ℓ = 0$).  We consider $(2,4,97)$-Steiner system (aka $2$-$(97,4,1)$
	design), $(2,3,49)$-Steiner system (aka $2$-$(49,3,1)$ design), complete
	graph on $17$ vertices, and P-BEST [Shental et al.\ 2020].  They all have
	code rate $N/T = 8$.  We plot their ROC curves.
\end{frame}

\pgfplotstableread{
	183fp  183sen   15fp   15sen    61fp   61sen
	0.65   91.622   3.018  96.277   1.182  86.561
	0.65   91.622   3.019  96.287   1.182  86.583
	0.656  91.633   3.052  96.356   1.244  86.87 
	0.679  91.679   3.12   96.477   1.365  87.367
	0.727  91.784   3.227  96.664   1.54   88.01 
	0.81   91.995   3.37   96.865   1.766  88.772
	0.931  92.29    3.552  97.125   2.039  89.643
	1.095  92.696   3.775  97.407   2.351  90.582
	1.304  93.19    4.028  97.709   2.709  91.595
	1.564  93.803   4.326  98.038   3.092  92.62 
	1.875  94.507   4.657  98.358   3.508  93.683
	2.234  95.298   5.015  98.689   3.946  94.744
	2.643  96.161   5.405  98.98    4.42   95.845
	3.099  97.062   5.822  99.282   4.921  96.933
	3.611  98.033   6.272  99.567   5.441  98.   
	4.169  99.029   6.744  99.792   5.982  99.028
	4.775 100.      7.236 100.      6.546 100.   
}\tablehyper
\begin{frame}[t]
	\leftskip10cm
	\begin{tikzpicture}[ROC]
		\begin{axis}[RoC]
			\addplot table[x=183fp,y=183sen]{\tablehyper};\addlegendentry{$(2,3,183)$-S}
			\addplot table[x=15fp,y=15sen]{\tablehyper};\addlegendentry{$K^{(3)}_{15}$}
			\addplot table[x=61fp,y=61sen]{\tablehyper};\addlegendentry{$K_{61}$}
		\end{axis}
		\global\advance\accumulatenumplots3
	\end{tikzpicture}
	\par
	Assume prevalence rate $p = 0.5\%$, uniform Ct values on $[16,32]$, and no
	delay ($ℓ = 0$).  We consider Kirkman triple system on $183$ vertices,
	complete $3$-uniform hypergraph on $15$ vertices, and complete graph on $61$
	vertices.  The first two have code rate $N/T = 30+1/3$; the last one has
	code rate $N/T = 30$.  We plot their ROC curves.
\end{frame}

\fontsize{12pt}{12pt}\selectfont

\begin{frame}\centering
	\huge\alert{Appendix}
	\par
	\Huge Comparison of GT Models
\end{frame}

\pgfplotstableread{
	Regime             Reading                       Remixing            
	Binary             {Negative, Positive}          $†Neg†∨†Pos†=†Pos†$ 
	Tropical           $2^{-∞},…,2^{-40},…,2^{-0}$   $\min(30,15)=15$    
	Semiquantitative   $[0,3),[3,6),[6,9),…$         $[0,3)+[3,6)=[3,9)$ 
	Quantitative       $0,1,2,3,4,5,…$               $8+9=17$            
}\tablequant
\begin{frame}\centering
	Four ways to quantify and combine test outputs.  Binary tests output
	``negative'' or ``positive''; combining samples means logical OR.
	Quantitative tests output numbers; combining samples means addition.
	The other two regimes lie in between.
	\par
	\def\arraystretch{2}
	\pgfplotstabletypeset[
		every head row/.style={before row=\toprule,after row=\midrule},
		every last row/.style={after row=\bottomrule},string type,
	]\tablequant
\end{frame}

\begin{frame}\centering
	\huge\alert{Appendix}
	\par
	\Huge Compressed Main Results
\end{frame}

\pgfplotstableread{
	Rounds        tests          people    patients   {max delay}
	$1$           $㏒_{ℓ+1}(N)$   $N$       $1$        $ℓ$
	$1$           $2√N$           $N$       $2$        $√N$
	$1$           $㏒₂(N)$        $N$       $2$        $3㏒₂(N)$
	$2$           $4$             $N$       $2$        $∞$
	$3D+1$        $3D+1$          $N$       $D$        $N$
	$4D㏒_ℓ(N)$   $4D㏒_ℓ(N)$      $N$       $D$        $ℓ$
}\tableresult
\begin{frame}\centering
	Main results of this work.  Round-$1$ testing schemes are non-adaptive.  The
	row with $2√N$ tests is special in that every person participates in only
	two tests.
	\par
	\def\arraystretch{2}
	\pgfplotstabletypeset[
		every head row/.style={before row=\toprule,after row=\midrule},
		every last row/.style={after row=\bottomrule},string type,
	]\tableresult
\end{frame}

\end{document}